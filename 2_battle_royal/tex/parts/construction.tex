\pagebreak

\section{Построение модели}
	Рассматривается ограниченная зона, где обитают два вида животных: травоядные (жертвы) и хищники. Предполагается, что животные не покидают территорию, у травоядных достаточно пищи, а хищники питаются исключительно жертвами.

	Построим математическую модель взаимодействия двух популяций в зависимости от времени. 

	Пусть \( x(t) \) -- популяция травоядных, \( y(t) \) -- популяция хищников на момент времени \( t \). 

	При отсутствии взаимодействия, то есть внешней угрозы от хищников, травоядные продолжают размножаться с коэффициентом рождаемости \( a > 0 \). Изменение их популяции можно выразить как
	\begin{align*}
		\dfrac{dx}{dt} = ax.
	\end{align*}

	Пока хищники не охотятся на жертв, они вымирают с неким коэффициентом убыли \( b > 0 \)
	\begin{align*}
		\dfrac{dy}{dt} = -bx.
	\end{align*}

	При взаимодействии каждый хищник поедает жертв с коэффициентом \( с \):
	\begin{align*}
		\dfrac{dx}{dt} = (a - сy)x.
	\end{align*}

	Сытые хищники способны к размножению с коэффициентом \( d \):
	\begin{align*}
		\dfrac{dx}{dt} = (dx - b)y.
	\end{align*}


	С учётом этих факторов можно составить следующую математическую модель борьбы популяций:
	\begin{equation}
		\begin{cases}
			\dfrac{dx}{dt} = (a - cy)x, \\
			\dfrac{dy}{dt} = (dx - b)y, \\
			x(0) = x_0, \\
			y(0) = y_0.
		\end{cases}
	\end{equation}


\pagebreak