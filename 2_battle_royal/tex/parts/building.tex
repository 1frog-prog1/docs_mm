\pagebreak

\section{Вычислительные эксперименты}
	Для численного нахождения решения системы дифференциальных уравнений с заданными параметрами используем метод Рунге-Кутты и построим графики решений \( x(t), y(t) \), а также фазовую траекторию \( F(x, y) \), отображающую взаимосвязь двух популяций.

	Проведём численные эксперименты при параметрах:
	\begin{equation*}
		a = 7, b = 4, c = 0.7, d = 0.5.
	\end{equation*}

	Было проведено 6 численных экспериментах с разными начальными условиями \( \left(x_0, y_0 \right) \).

	В точках покоя \( (0, 0) \) и \( \left( \frac{b}{d}, \frac{a}{c} \right) \) функция не изменяется, то есь значения \( x(t), y(t) \) остаются постоянными вне зависимости от времени.

	\graphic{pics/building_graphics/0_population_plots.png}{Графики популяций травоядных и хищников в точке \( (0, 0) \)}{9}

	\graphic{pics/building_graphics/1_population_plots.png}{Графики популяций травоядных и хищников в точке \( \left( \frac{b}{d}, \frac{a}{c} \right) \)}{9}

	В других точках, на некотором отдалении от точки \( \left( \frac{b}{d}, \frac{a}{c} \right) \) функции \( x(t), y(t) \) цикличны.

	\graphic{pics/building_graphics/population_plots.png}{Графики популяций травоядных и хищников в отдалении от \( \left( \frac{b}{d}, \frac{a}{c} \right) \)}{18}

	Эту цикличность можно лучше увидеть на фазовой плоскости \( \left( x, y \right) \):

	\graphic{pics/building_graphics/population_interaction.png}{ График взаимодействия популяций }{15}

	Как мы и проанализировали, в точках \( (0, 0) \) и  \( \left(\frac{b}{d}, \frac{a}{c} \right) \) количество жертв и хищников не изменяется со временем. И так как точка \( \left(\frac{b}{d}, \frac{a}{c} \right) \) является центром, из-за чего вокруг неё образовались циклы.


\pagebreak