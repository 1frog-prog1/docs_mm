\pagebreak

\section{Введение}
	Математические модели взаимодействия между хищниками и их жертвами представляют собой важный инструмент для изучения динамики экосистем. Модель "хищник-жертва" является классическим примером таких моделей, которая позволяет описать взаимосвязь между популяциями хищников и жертв в природной среде. В данном отчете мы построим математическую модель этой системы и исследуем ее поведение в различных условиях.

	Применение модели "хищник-жертва" не ограничивается только биологическими науками. Она также находит широкое применение в экологии, управлении ресурсами, а также в экономике и социологии для анализа динамики взаимодействия между различными группами или процессов. Понимание основных принципов этой модели может помочь в прогнозировании изменений в популяциях и разработке стратегий устойчивого управления ресурсами.

	Таким образом, изучение модели "хищник-жертва" имеет не только теоретическое значение, но и практическую важность для понимания сложных систем и развития эффективных стратегий управления ими.

\pagebreak