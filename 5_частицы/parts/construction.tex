\pagebreak

\section{Построение модели}
	Для изучения процессов распространения применяется уравнение переноса, которое представляет собой дифференциальное уравнение в частных производных, описывающее изменения скалярной величины в пространстве и времени. Уравнение переноса записывается следующим образом:
	\begin{equation}
		\dfrac{dC}{dt} + u(x, y)\dfrac{dC}{dx} + v(x, y)\dfrac{dC}{dy} = 0,
	\end{equation}
	где
	\begin{itemize}
		\item \( t \) -- время,
		\item \( x, y \) -- координата в пространстве,
		\item \( u, v \) -- компонента скорости перемещения по \( x, y \) соответственно.
	\end{itemize}

	Задано поле концентрации в начальный момент:
	\begin{equation}
		C(0, x, y) = C_0(x, y).
	\end{equation}

	Поле скорости зададим через функцию тока \( \psi(x, y) \). Скорость течения по осям \( OX, OY \):
	\begin{equation}
		\begin{cases}
			u(x, y) = & - \dfrac{d \psi}{d y}, \\
			v(x, y) = & \dfrac{d \psi}{d x}.
		\end{cases}
	\end{equation}
	

\pagebreak