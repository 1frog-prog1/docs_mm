\pagebreak

\section{Введение}
    Электрические нагреватели широко применяются в различных областях, начиная от промышленности и заканчивая бытовыми устройствами. Создание математической модели электрического нагревателя играет важную роль в оптимизации процессов нагрева, повышении эффективности и безопасности его работы. Математическая модель позволяет предсказывать тепловые характеристики нагревателя в зависимости от различных параметров, таких как мощность тока, материал нагревательного элемента, его площадь и так далее.

	Применение математической модели электрического нагревателя может быть полезным в различных областях, включая промышленное производство, медицину, сельское хозяйство и научные исследования. Например, она может помочь оптимизировать процессы термической обработки материалов, контролировать температурные режимы в медицинских устройствах или разрабатывать новые методы обогрева в сельском хозяйстве.

	Важность разработки математической модели без терморегулятора заключается в возможности понимания динамики нагрева и охлаждения системы при изменении различных параметров. Это позволит нам определить оптимальные режимы работы нагревателя для конкретных задач.

	Создание математической модели электрического нагревателя с терморегулятором позволит нам более точно контролировать температурный режим работы устройства. Это крайне важно для предотвращения перегрева или недостаточного нагрева системы, что может привести к повреждению оборудования или снижению эффективности работы.

	Целью этой лабораторной работы является построение и анализ математических моделей электрического нагревателя без терморегулятора и с ним. С помощью построенной модели мы хотим изучить, как изменяется температура нагревателя с течением времени в зависимости от разных параметров.

\pagebreak