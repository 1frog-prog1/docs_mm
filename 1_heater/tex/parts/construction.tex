\pagebreak

\section{Построение модели}
	
	Для того чтобы понять, как меняется температура нагревателя со временем, мы должны разобраться, что именно влияет на этот процесс и какие законы описывают изменение температуры.

	В процессе взаимодействия частей замкнутой теплоизолированной системы между ними устанавливается состояние теплового равновесия.

	Следствием закона сохранения энергии для замкнутой теплоизолированной системы служит уравнение теплового баланса:
	\begin{equation}
		\Delta Q = \Delta Q_1 + \dots + \Delta Q_n. \label{q_balance}
	\end{equation}

	Рассмотрим процессы, которые участвуют при использовании электрического нагревателя и его нагревании в уравнении теплового баланса.

	\begin{enumerate}
		\item В процессе нагревания тела происходит изменение внутренней энергии \( \Delta Q_1 \) за счёт изменения температуры тела:
		\begin{equation}
			\Delta Q = mc \Delta T, \label{Q1}
		\end{equation} где \begin{itemize}
			\item \( m \left[\text{кг} \right] \) -- масса тела,
			\item \( c \left[\dfrac{\text{Дж}}{\text{кг} \cdot \text{К}} \right] \) -- удельная теплоемкость тела,
			\item \( \Delta T \left[\text{К} \right] \) -- изменение температуры тела.
		\end{itemize}

		\item При использовании электрического нагревателя, электрический ток протекает через специально созданный проводник, из-за чего происходит выделение теплоты. Оно измеряется по формуле:
		\begin{equation}
			\Delta Q_1 = P \Delta t,
		\end{equation} где \begin{itemize}
			\item \( P \left[\text{Вт} \right] \) -- мощность тока,
			\item \( \Delta t\left[\text{с} \right]  \) -- прошедшее время.
		\end{itemize}

		\item В процессе нагревания тела также происходит и его охлаждение за счёт температуры окружающей среды. Этот процесс может быть записан с помощью закона теплоотдачи Ньютона-Рихмана:
		\begin{equation}
			\Delta Q_2 = -kS \left( T - T_{out} \right) \Delta t,
		\end{equation} где \begin{itemize}
			\item \( k \left[\dfrac{\text{Вт}}{\text{м}^2 \cdot \text{К}} \right] \) -- коэффициент теплоотдачи,
			\item \( S \left[\text{м}^2 \right] \) -- площадь поверхности тела,
			\item \( T \left[\text{К} \right] \) -- температура нагреваемого тела,
			\item \( T_{out} \left[\text{К} \right] \) -- температура окружающей среды,
			\item \( \Delta t \left[\text{с} \right] \) -- прошедшее время.
		\end{itemize}

		\item Все тела, температура которых выше абсолютного нуля, испускают тепловое излучение, расчитываемое по формуле:
			\begin{equation}
				\Delta Q_3 = - \sigma S \left( T^4 - T_{out}^4 \right) \Delta t,
			\end{equation}
			где \begin{itemize}
				\item  \( \Delta Q \) -- количество выделившейся теплоты тела,
				\item \( S  \left[\text{м}^2 \right] \) -- плозадь поверхности,
				\item \( T \left[\text{К} \right] \) -- температура нагревателя в кельвинах,
				\item \( T_0 \left[\text{К} \right] \) -- температура окружающей среды в кельвинах,
				\item \( \sigma \) -- постоянная Стефана-Больцмана, которая равна \(5.67 \cdot 10^-8 \dfrac{\text{Вт}}{\text{м}^2 \cdot \text{К}^4} \),
				\item \( \Delta t \) -- прошедшее время.
			\end{itemize}
	\end{enumerate}

	Таким образом, в заданных ранее условиях рассмотрения модели, мы получаем следующее уравнение теплового баланса для электрического нагревателя:
	\begin{equation}
		mc \Delta T = P \Delta t -kS \left( T - T_{out} \right) \Delta t - \sigma S \left( T^4 - T_{out}^4 \right) \Delta t.
		\label{final_eq}
	\end{equation}

	Наша поставленная задача состоит в том, чтобы построить функцию изменения температуры нагревателя в зависимости от времени. Потому выразим из полученной формуле (\ref{final_eq}) дифференциальное уравнение относительно \( \dfrac{dT}{dt} \).
	Для этого поделим уравнение (\ref{final_eq}) на \( mc \Delta t \)
	\begin{equation*}
		\dfrac{\Delta T}{\Delta t}  = \dfrac{P -kS \left( T - T_{out} \right) - \sigma S \left( T^4 - T_{out}^4 \right)}{mc},
	\end{equation*}

	и устремим \( \Delta t \rightarrow 0 \)
	\begin{equation*}
		\dfrac{d T}{d t}  = \dfrac{P -kS \left( T - T_{out} \right) - \sigma S \left( T^4 - T_{out}^4 \right)}{mc}.
	\end{equation*}

	Таким образом, мы построили дифференциальное уравнение температуры \( T \) электрического нагревателя в зависимости от времени \( t \). Решая его можно найти искомую функцию \( T(t) \).

	Добавив начальные условия, мы получаем математическую модель электрического нагревателя без терморегулятора:

	\[
		\begin{cases}
			\dfrac{d T}{dt} = \dfrac{P - kS \left(T - T_{out} \right) - \sigma S \left(T^4 - T_{out}^4 \right)}{mc}, \\
			T(0) = T_{out}.
		\end{cases}
	\]

	Для построения математической модели с терморегулятором, необходимо ввести функцию \( H(T) \), которая управляет температурой нагревателя, прерывая поступления тока при превышении верхней границы \( T_{up} \) и возобновляя её при опускании температуры ниже нижней границы \( T_{down} \).
	\[
		\begin{cases}
			\dfrac{d T}{dt} = \dfrac{H(T) \cdot P - kS \left(T - T_{out} \right) - \sigma S \left(T^4 - T_{out}^4 \right)}{mc}, \\
			T(0) = T_{out}, 
		\end{cases} 
		~ H(T) = \begin{cases}
			0, \, &T > T_{up}, \\
			1, \, &T < T_{down}.
		\end{cases}
	\]

	Таким образом, мы построили математические модели, описывающие изменение температуры электронагревателя с учетом и без учета работы терморегулятора в зависимости от времени.

	\section{Математические модели электрического нагревателя}
		Математическая модель без терморегулятора
			\[
				\begin{cases}
					\dfrac{d T}{dt} = \dfrac{P - kS \left(T - T_{out} \right) - \sigma S \left(T^4 - T_{out}^4 \right)}{mc}, \\
					T(0) = T_a.
				\end{cases}
			\]

		Математическая модель с терморегулятором
			\[
				\begin{cases}
					\dfrac{d T}{dt} = \dfrac{H(T) \cdot P - kS \left(T - T_{out} \right) - \sigma S \left(T^4 - T_{out}^4 \right)}{mc}, \\
					T(0) = T_a, 
				\end{cases}
			\] где 
			\(
				H(T) = \begin{cases}
					0, \, &T > T_{up}, \\
					1, \, &T < T_{down}.
				\end{cases} 
			\)

		\subsection*{Обозначения}
			\begin{itemize}
				\item \( T \left[\text{К} \right] \) -- температура нагревателя в Кельвинах
				\item \(t \left[\text{с} \right] \) -- время в секундах
				\item \( P \left[\text{Вт} \right] \) -- мощность тока
				\item \( S \left[\text{м}^2 \right] \) -- площадь поверхности нагревателя
				\item \( T_0 \left[\text{К} \right] \) -- температура внешней среды
				\item \( c \left[\dfrac{\text{Дж}}{\text{кг} \cdot \text{К}} \right] \) -- удельная теплоемкость материала
				\item \( m \left[\text{кг} \right] \) -- масса тела
				\item \( k \left[\dfrac{\text{Вт}}{\text{м}^2 \cdot \text{К}} \right]\) -- коэффициент теплопередачи
				\item \( \sigma \left[\dfrac{\text{Вт}}{\text{м}^2 \cdot \text{К}^4} \right] \) -- постоянная Стефана-Больцмана
			\end{itemize}

		Построенные модели представляют из себя задачу Коши для обыкновенного уравнения 1-го порядка, нелинейное, разрешенное относительно производной.


\pagebreak