\pagebreak

\section{Реализация модели}
	Математическая модель представляет собой задачу Коши для обыкновенного дифференциального уравнения 1-го порядка, разрешенное относительно производной. Для решения можно использовать метод Рунге-Кутты четвертого порядка, который имеет погрешность порядка \(O(h^4)\), где \( h \) -- шаг метода.

	Построим зависимость температуры от времени $T(t)$ при следующих выбранных параметрах:
	\begin{itemize}
		\item $P = 800\text{Вт}$;
		\item $S = 0.5 \text{м}^2$;
		\item $T_0 = 298 \text{К}$;
		\item $c = 1460 \dfrac{\text{Дж}}{\text{кг} \cdot \text{К}}$;
		\item $m = 1 \text{кг}$;
		\item Шаг, с которым мы будем замерять время $h = 0.1$;
	\end{itemize}

	\subsection{Нагреватель без терморегулятора}
		\graphic{pics/building_graphics/T_t.png}{График нагревания без терморегулятора}{12}

		\lstinputlisting[caption=Вычисление значений методом Рунге-Кутта без терморегулятора, style=code_style]{src/runge_kutta.py}

	\subsection{Нагреватель с терморегулятором}
		\graphic{pics/building_graphics/T_t_relle.png}{График нагревания с терморегулятором}{16}

		\lstinputlisting[caption=Вычисление значений методом Рунге-Кутта с терморегулятором, style=code_style]{src/runge_kutta_relle.py}


\section{Численные эксперименты}
	Проведём серию численных экспериментов, варьируя каждый параметр по отдельности, чтобы выяснить, какие влияние они оказывают на математическую модель.

	\graphic{pics/building_graphics/T_t_massa.png}{Влияние массы на модель}{12}
	Из этого графика можно сделать вывод, что масса нагревателя не влияет на максимальную температуру, но влияет на скорость, с которой она достигается.

	\graphic{pics/building_graphics/T_t_P.png}{Влияние мощности на модель}{12}
	Наблюдение показывает, что чем больше мощность тока, тем выше температура, которую нагреватель может достичь. Это логично, поскольку большая мощность обеспечивает больше энергии для нагрева.

	\graphic{pics/building_graphics/T_t_S.png}{Влияние площади на модель}{12}
	Отсюда видно, что площадь поверхности нагревателя влияет обратно пропорционально на максимальную температуру. Это может быть связано с тем, что более большая поверхность может эффективнее отводить тепло, что может снижать максимально достижимую температуру.


\pagebreak