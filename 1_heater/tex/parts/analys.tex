\pagebreak

\section{Анализ модели}

	\subsection{Максимальная температура}
	В начальный момент времени разогрева нагревателя \( T = T_{out} \), отсюда:
	\[ \dfrac{dT}{dt} = \dfrac{P}{cm} > 0.\]
	Это обозначает, что температура будет увеличиваться. Поскольку производная \( \dfrac{dT}{dt} \) представляет из себя непрерывную функцию, то изменение температуры будет плавным. Со временем отрицательные компоненты будут возрастать. что приведёт к замедлению роста функции \( T(t) \). В конечном итоге мощность P и отрицательные компоненты сбалансируют друг друга, достигнув равновесия:
	\[ \dfrac{d T}{dt} = \dfrac{P - kS \left(T - T_{out} \right) - \sigma S \left(T^4 - T_{out}^4 \right)}{mc} = 0.\]
	После чего температура больше не будет увеличиваться, достигнув максимального значения. 

	Значение максимальной температуры можно узнать, решив уравнение:
	\[ P - kS \left(T - T_{out} \right) - \sigma S \left(T^4 - T_{out}^4 \right) = 0 \]
	Соберём общие множители у степеней \( T \):
	\[ T^4 + \dfrac{k}{\sigma} T - \left(T_0^4 + \dfrac{kT_0 + \frac{P}{S}}{\sigma} \right) = 0.\]
	Это уравнение четвёртой степени, а значит оно имеет ровно 4 корня. Убедимся, что существует лишь одно допустимое решение этого уравнения, которое и будет является максимальной температурой. Для того, чтоб решение \( T^* \) являлось максимальным значением для задачи, необходимо:
	\begin{align}
		&T > 0, \\
		&T \in R. 
	\end{align} \label{conds}

	Для удобства переобзначим:
	\begin{align*}
		&a = \dfrac{k}{\sigma} > 0, \\
		&b = \left(T_0^4 + \dfrac{kT_0 + \frac{P}{S}}{\sigma} \right), \\
		T^4 + aT - b = 0.
	\end{align*}



\pagebreak