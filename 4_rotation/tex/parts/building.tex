\pagebreak

\section{Вычислительные эксперименты}

	\subsection{Метод для вычисления}
		Для вычисления решения был использован метод Рунге-Кутта. Он позволяет вычислять решения с погрешностью \( O(h^4) \), где \( h \) -- заданый шаг.

		\lstinputlisting[caption=Код метода Рунге-Кутта, style=code_style]{src/runge_kutta.py}

	\subsection{Результаты}
		Рассмотрим траектории, начинающиеся из точки \( (8, 4) \) c разными имеющимися начальными скоростями.

		\graphic{pics/building_graphics/10.png}{Рисунок при угловой скорости \( w = 1.5 \)}{18}

		На численном эксперименте подтвердили гипотезу о том, что траектории движения принимают вид окружности. Как и то, что чем больше начальные скорости \( x'_0, y'_0 \) по модулю, тем больше радиус окружности.

		Закон сохранения энергии не выполняется точно, но это можно связать с погрешностью численного метода. Так как при уменьшении шага \( h \) уменьшается и погрешность. В данном примере шаг \( h = 10^{-4} \), и полученная погрешность при таком шаге равна \( O(10^{-13}) \).

		Рассмотрим влияние угловой скорости \( w \) на траектории движения. В прошлом примере \( w  = 1.5 \), 
		теперь же уменьшим ее до \( w = \dfrac{\pi}{6} \) и рассмотрим траектории для тех же начальных точек.

		\graphic{pics/building_graphics/100.png}{Рисунок при угловой скорости \( w = \frac{\pi}{6} \)}{18}

		Как видим, при уменьшении угловой скорости \( w \) радиусы увеличиваются. 

\pagebreak