\pagebreak

\section{Анализ модели}
	\begin{equation*}
		\begin{cases}
			& \dfrac{d^2 x}{dt^2} = 2w \dfrac{dx}{dt}, \\
			&\dfrac{d^2 y}{dt^2} = -2w \dfrac{dy}{dt}, \\
			& x(0) = x_0, \quad x'(0) = x'_0, \\
			& y(0) = y_0, \quad y'(0) = y'_0.
		\end{cases}
	\end{equation*}

	Разделим первую строку на вторую и получим:
	\begin{equation}
		\dfrac{d^2 x}{d^2y} =  -\dfrac{dy}{dx} \rightarrow d^2x \cdot dx + d^2 y \cdot dy = 0.
	\end{equation}

	Это уравнение можно проинтегрировать:
	\begin{equation} \label{res}
		\int dx \cdot d^2x + \int d y \cdot d^2 y = 0 \rightarrow dx^2 + dy^2 = C = x'^2_0 + y'^2_0.
	\end{equation}

	Таким образом получили в левой части сумму проекций кинетической инергии \( E_k \) и что она остается постоянной, следуя из правой. Так и должно быть, так как наша система замкнутая, потому и закон сохранения энергии должен действовать.

	Отсюда же можно сделать вывод, что траектории движения будут принимать вид окружности, так как уравнение (\ref{res} ) выглядит как уравнение окружности. Из уравнения (\ref{res} ) также можно сделать вывод, чем чем больше начальные значения скорости \( x'_0, y'_0 \), тем больше радиус у полученной окружности.

	По полученному закону можно вывести соотношение, по которому будем проверять точность численного решения.

	\begin{equation}
		dx^2 + dy^2 - x'_0^2 - y'_0^2 = 0.
	\end{equation}


\pagebreak