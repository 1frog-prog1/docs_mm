\pagebreak

\section{Построение модели}
	Перед началом построения сделаем некоторые предположения
	\begin{itemize}
		\item рассматриваем движение тела относительно вращающейся системы отчета с системой координат \( (x, y) \),
		\item вращение происходит с постоянной угловой скоростью \( w \),
		\item тело является материальной точкой с массой \( m \),
		\item не рассматриваем силу трения.
	\end{itemize}

	Поскольку система отсчёта неинерцианальна, то на тела действует сила  \( F_u \), равная
	\begin{equation}
		\vec{F_u} = - m \dfrac{d \vec v}{dt},
	\end{equation}
	где \( m \) -- масса, \( \vec v = \left(v_x, v_y \right) \) --  вектор скорости тела. 

	При движении тела относительно вращающейся системы известно, что появляется сила Кориолиса, вычисляемая по формуле:
	\begin{equation}
		\bar F_k = 2m \cdot [\vec v, \vec w].
	\end{equation}

	Приравнивая между собой силы получается:
	\begin{equation}
		- m \dfrac{d \vec v}{dt} = 2m \cdot [\vec v, \vec w].
	\end{equation}

	Сократив уравнение на \( m \) и рассматривая изменение положения по каждой компоненте, получаем систему линейных дифферециальных уравнений \rimint{2} порядка:
	\begin{equation}
		\begin{cases}
			&\dfrac{d^2 x}{dt^2} = 2w \dfrac{dx}{dt}, \\
			&\dfrac{d^2 y}{dt^2} = -2w \dfrac{dy}{dt}.
		\end{cases}
	\end{equation}

	Добавим начальные условия и получим математическую модель, описывающую в неинерциальной системе отсчета.

	\begin{equation}
		\begin{cases}
			&\dfrac{d^2 x}{dt^2} = 2w \dfrac{dx}{dt}, \\
			&\dfrac{d^2 y}{dt^2} = -2w \dfrac{dy}{dt}, \\
			& x(0) = x_0, \quad x'(0) = x'_0, \\
			& y(0) = y_0, \quad y'(0) = y'_0.
		\end{cases}
	\end{equation}

\pagebreak