\pagebreak

\section{Введение}
	До сих пор мы имели дело с инерциальными системами отсчета, то есть системами, в которых выполняются законы Ньютона. Системы отсчета, которые движутся относительно инерциальных систем с ускорением, называются неинерциальными. В них законы Ньютона в обычном виде применять нельзя, требуется введение специальных поправок — сил инерции.

	Движение на вращающемся диске является примером неинерциальной системы отсчета. На вращающемся диске тело, двигаясь относительно диска, подвергается центробежным силам. Центробежные силы возникают из-за ускорения тела относительно центра вращения. Эти силы изменяют траекторию движения тела и делают движение на вращающемся диске отличным от равномерного прямолинейного движения.

	Понимание этого явления позволяет углубить знания о физике и механике, а также применить их в различных практических ситуациях.


\pagebreak