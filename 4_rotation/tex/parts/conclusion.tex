\pagebreak
	
	\section{Заключение}
		Была построена математическая модель во вращающейся система отсчета. Она представляла из себя систему двух дифференциальных уравнения \rimint{2} порядка. При анализе было выявлено, что траектории движения выглядят как окружности. И чем больше по модулю начальные скорости тела, тем больше радиус окружности движения. Были построены численные решения из одной точки с разными начальными скоростями, чтоб подтвердить эту гипотезу, а также увидеть зависимость радиуса траекторий от угловой скорости вращения \( w \): чем меньше \( w \), тем больше радиус окружности.

\pagebreak