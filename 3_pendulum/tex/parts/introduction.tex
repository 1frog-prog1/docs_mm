\pagebreak

\section{Введение}
	В нашем мире происходит множество явлений, одно из которых — колебания математического маятника. 

	Математический маятник служит простейшей моделью физического тела, совершающего колебания, не учитывая распределение массы. Однако реальный физический маятник при малых амплитудах колеблется так же, как и математический.

	Изучение различных моделей маятника позволяет лучше понять особенности колебательных систем и их поведение в различных условиях, что имеет важное значение для практических применений и теоретических исследований в области физики и механики.

	Модель математического маятника будет реализована в нескольких вариантах:
	\begin{itemize}
		\item Без учёта трения и вынужденнных колебаний;
		\item Только с учётом трения;
		\item Только с учётом вынужденных колебаний;
		\item С учётом трения и вынужденных колебаний;
	\end{itemize}

	А также, для воспроизведения математической модели явления резонанса, проведём серию экспериментов для выявления закономерностей.

\pagebreak