\pagebreak

\section{Построение модели}
	Математический маятник представляет собой идеальную механическую систему, где на одном конце невесомой и нерасстягивающейся нити находится материальная точка массой \( m \), а другой конец нити (стержня) обычно неподвижен. 

	Математический маятник сможет колебаться только в одной плоскости вдоль выделенного горизонтального направления. При колебаниях в одной плоскости маятник движется по дуге окружности радиуса \( l \). Тогда движение маятника можно описать через угол отклонения от вертикальной оси \( \alpha \).

	\graphic{pics/pendulum_schema.jpg}{Рисунок математического маятника}{7}

	\subsection{Нелинейная математическая модель без учёта внешних сил}

		Воспользумся уравнением моментов для материальной точки:
		\begin{equation}
			J \dfrac{d^2 \alpha}{dt^2} = M, \label{moments_formula}
		\end{equation}
		где \( \alpha \) -- угол наклона маятника в настоящий момент, \( J \) -- момент инерции относительно оси:
		\begin{equation}
			J = m \cdot l^2. \label{J_formula}
		\end{equation}

		Если тело не находится в положении равновесия, то на него действует момент силы \( M \):
		\begin{equation}
			M = F \cdot l,
		\end{equation}
		где  \( l \) -- длина нити, \( F \) -- возвращающая сила, проекция силы тяжести на ось движения маятника. То есть момент силы рассчитывается по формуле
		\begin{equation}
			M = -mg \cdot l \sin \alpha. \label{M_formula}
		\end{equation}

		Подставив формулы (\ref{J_formula}) и (\ref{M_formula}) в уравнение моментов (\ref{moments_formula}), получим:
		\begin{equation}
			m \cdot l^2 \dfrac{d^2 \alpha}{dt^2} = -mg \cdot l\sin \alpha \Rightarrow \dfrac{d^2 \alpha}{dt^2} + \dfrac{g}{l} \sin \alpha = 0.
		\end{equation}

		Получили нелинейное дифференциальное уравнение \rimint{2} порядка, описывающее угол отклонения маятника в зависимости от времени.

		Обозначив за \( w^2 = \dfrac{g}{l} \) и добавив начальные условия, получим неинейную математическую модель без учёта трения и вынужденных колебаний.

		\begin{equation}
			\begin{cases}
				& \dfrac{d^2 \alpha}{dt^2} + w^2 \sin \alpha = 0, \\
				& \alpha(0) = \alpha_0, \\
				& \alpha'(0) = \alpha_1.
			\end{cases} \label{nonlinear_model}
		\end{equation}

	\subsection{Линейная математическая модель без учёта внешних сил}

		Если заранее известно, что угол значения принимает небольшие значения, тогда можно заменить \( \sin \alpha \approx \alpha \) и получить линейное дифференциальное уравнение, которое будет иметь небольшую погрешность от решения (\ref{nonlinear_model}):
		\begin{equation}
			\begin{cases}
				& \dfrac{d^2 \alpha}{dt^2} + w^2  \alpha = 0, \\
				& \alpha(0) = \alpha_0, \\
				& \alpha'(0) = \alpha_1.
			\end{cases}
		\end{equation}

	
	Рассмотрим модель маятника с оказываемыми на него внешними силами.

	\subsection{Математическая модель с учётом трения}

		Одной из внешних сил является трение, из-за которого колебания со временем будут постепенно затухать. Сила трения зависит от скорости \( \dfrac{d \alpha}{dt} \) и от некоторого коэффициента трения \( k > 0 \). При учёте действия трения на маятник, получаем математическю модель движения маятника с учётом трения.

		\begin{equation}
			\begin{cases}
				& \dfrac{d^2 \alpha}{dt^2} +  k \dfrac{d \alpha}{dt} + w^2 \sin \alpha = 0, \\
				& \alpha(0) = \alpha_0, \\
				& \alpha'(0) = \alpha_1.
			\end{cases} 
		\end{equation}

	\subsection{Математическая модель с вынужденными колебаниями}
		Также можем рассмотреть внешнее воздействие в виде вынужденных колебаний. Они задаются через амплитуду вынуждающей силы \( A_f \) и частоту колебаний \( w_f \):
		\begin{equation}
			F_o = A_f \cdot \cos(w_f t).
		\end{equation}

		Математическая модель движения маятника только при действии вынужденных колебаний выглядит следующим образом:
		\begin{equation}
			\begin{cases}
				& \dfrac{d^2 \alpha}{dt^2} + w^2 \sin \alpha = A_f \cdot \cos(w_f t), \\
				& \alpha(0) = \alpha_0, \\
				& \alpha'(0) = \alpha_1.
			\end{cases}
		\end{equation}

	Если рассматривать движения маятника при действии и силы трения, и вынужденных колебаний, то получим:
	\begin{equation}
			\begin{cases}
				& \dfrac{d^2 \alpha}{dt^2} +  k \dfrac{d \alpha}{dt} + w^2  \alpha = A_f \cdot \cos(w_f t), \\
				& \alpha(0) = \alpha_0, \\
				& \alpha'(0) = \alpha_1.
			\end{cases}
		\end{equation}


\pagebreak