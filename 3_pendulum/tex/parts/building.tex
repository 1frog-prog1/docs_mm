\pagebreak

\section{Численные эксперименты}

	\subsection{Метод решения}
		Модели представляют из себя краевые задачи для дифференциального уравнения 2-го порядка, потому их решения можно найти численно с помощью метода Рунге-Кутта, который обеспечивает погрешность \( O(h^4) \).

	\subsection{Программа}
		\lstinputlisting[caption=Код метода Рунге-Кутта, style=code_style]{code/runge_kutta.py}

	\subsection{Результаты}
		\subsubsection{Модель без учёта внешних сил}

			Проанализируем поведение решения точной, нелинейной, модели и приближенной, линейной, модели. 

			\graphic{pics/pendulum_examples/simple/pi_6.png}{\( \alpha(0) = \frac{\pi}{6} \)}{7}

			При небольших колебаниях в самом начале решение линейной модели почти идеально повторяет поведения точного решения. Но с увеличением времени видно, что нарастает ошибка.

			Чем больше угол колебания \( \alpha \), тем быстрее нарастает ошибка. Это отлично можно заметить из следующего графика.

			\graphic{pics/pendulum_examples/simple/pi_3.png}{\( \alpha(0) = \frac{\pi}{3} \)}{7}

			В начале приближение почти идеально повторяет поведение точного решения. Но уже на \( t = 10 \) разница составляет аж половину амплитуды колебаний.

		\subsubsection{Модель с учётом трения}

			При учитывании трения решение постепенно сходится к нулю, как и было сказано в анализе модели.

			\graphic{pics/pendulum_examples/friction/k_03.png}{\( k = 0.03 \)}{7}

			Чем меньше коэффицициент трения, тем медленнее происходит затухание колебаний.

			\graphic{pics/pendulum_examples/friction/k_006.png}{\( k = 0.006 \)}{7}

		\subsubsection{Модель с учётом гармонических колебаний}

			\graphic{pics/pendulum_examples/pendulums/0.4.png}{\( A_f = 1, w_f = 0.5 \)}{7}
			Отсюда можно заметить, что при вынужденных колебаний происходит смещение угла \( \alpha \). Причем изменение угла не циклично и на фазовом портрете не возвращается в ту же точку.

			Но ситуация иная, если частота вынужденных колебаний кратна частоте колебаний маятника. Тогда поведение маятника циклично является цикличным, что можно увидеть по следующим графикам.

			\graphic{pics/pendulum_examples/pendulums/1.4w.png}{\( A_f = 4, w_f = 1.4w \)}{7}

			При совпадении частот маятника и вынужденных колебаний происходит постоянное увеличение угла отклонения маятника. Это явление называется резонансом.

			\graphic{pics/pendulum_examples/pendulums/w.png}{\( A_f = 4, w_f = w \)}{7}

		\subsubsection{Модель с учётом гармонических колебаний и трения}

			Если на маятник одновременно действует сила трения и вынужденные колебания, то получается интересная картина. 

			При непропорциональных вынужденных колебаниях, система некоторе время будет нециклично сходиться.
			\graphic{pics/pendulum_examples/pendulums_fric/2.6.png}{\( w_f = 2.6 \)}{18}

			Но после принимает вид почти цикличных колебаний некой другой амплитуды и частоты. "Почти цикличные" они потому, что всо временем амплитуда все же растет, пусть и очень медленно.

			\graphic{pics/pendulum_examples/pendulums_fric/2.6_end.png}{Фазовая плоскость под конец наблюдения}{7}

			С трением и вынужденными колебаниями пропорциональной частоты получается, что со временем движение маятника становится цикличным.

			\graphic{pics/pendulum_examples/pendulums_fric/0.5.png}{\( w_f = 0.05w \)}{7}

			А вот при совпадении частот амплитуда колебаний растёт.

			\graphic{pics/pendulum_examples/pendulums_fric/w.png}{\( w_f = w \)}{7}

			Но спустя некоторое время, рост амплитуды сильно замедляется и становится почти цикличным.

			\graphic{pics/pendulum_examples/pendulums_fric/w_alot_end.png}{\( w_f = w \)}{7}

		\subsection{Резонанс}
			Как было замечено из рисунков (8) и (11), при совпадении частот возникает такое физическое явление, как резонанс. 

			\graphic{pics/pendulum_examples/rezonans.png}{Рисунок зависимости амплитуды от частоты вынужденных колебаний }{10}

			


\pagebreak